\documentclass[11pt,a4paper]{article}

\usepackage[slovene]{babel}
\usepackage[utf8x]{inputenc}
\usepackage{graphicx}

\pagestyle{plain}

\begin{document}
\title{Poročilo pri predmetu \\
Analiza podatkov s programom R}
\author{Žana Štekovič}
\maketitle

\section{Izbira teme}
Temo sem si izbrala na podlagi trenutnih situacij. Predvsem kaj bi me zanimalo in bi z veseljem raziskovala o tem.
Tema je IZOBRAZBA SLOVENIJE.

\section{Obdelava, uvoz in čiščenje podatkov}
Sama obdelava podatkov je izgledala tako, da sem iz spletne strani podatke shranila v csv obliki, potem pa sem vse nepotrebne podatke zbrisala, saj jih nisem hotela v tabeli.
Obločila sem se za dve tabeli. Prva se nanaša na celo Slovenijo, in sicer je primerjava moških in žensk pri treh različnih izobrazbah in treh letih.
Druga tabela pa se deli na regije, pri tem da sta spola moški in ženske združena, saj me bo zanimalo v nadaljnem, v katerih koncih sloveniji je stopnja izobrazbe najvišja, in kako se ta z leti spreminja.
V datoteki za uvoz sem naredila dve funkciji, vsaka za eno tabelo.Prvo tabelo sem uvozila v csv obliki, drugo tabelo pa v html obliki. Iz druge tabele sem vzela podatke za leto 2011, 1. stopnja izobrazbe in naredila graf za vse regije, in pa za leto 2013, 1. stopnja izobrazbe za vse regije. Ker so te razlike načeloma ne tako zelo velike razlika med grafoma ni očitna.
V nadaljnih fazah bom naredila primerjavo tudi na drugačen način, kjer bo vidno, da se iz leta v leto viša.  Iz spletne strani sem uvozila tudi podatke v html obliki. V datoteki \verb|xml.r| je funkcija \verb|uvozi.poregijah|. V \verb|uvoz3.R| pa sem definirala novo funkcijo, katera se sklicuje na \verb|uvozi.poregijah|. To sem naredila zato, da je v glavnem programu vse urejeno in da ni preveč podatkov.

\includegraphics[width=\textwidth]{../slike/leto_2011.pdf}
\includegraphics[width=\textwidth]{../slike/leto_2013.pdf}

\section{Analiza in vizualizacija podatkov}
V tej fazi sem narisala zemljevid in ga shranila v pdf obliko. Funkcija, ki izriše zemljediv je shranjena v datoteki \verb|vizualizacija\vizualizacija.r|. Z zemljevidom sem želela prikazati primerjavo po regijah koliko ljudi ima visokošolsko izobrazbo 1. stopnje v vsaki regiji. Seveda je treba tukaj upoštevati tudi koliko je regija velika saj je s tem pogojeno tudi število prebivalcev in s tem več možnosti da se bodo ljudje odločili za študij. Ampak tudi če primerjamo s številom prebivalcem po regijah je še vedno največ ljudi z visokošolsko 1.stopnjo izobrazbe v Osrednji Sloveniji. Kar kmalu za njo sledi Podravska potem Savinjska in Gorenjska. Zemljevid ima zraven tudi tabelico s številkami, da si lahko tudi številčno predstavljamo koliko ljudi je to.
\includegraphics{../slike/zemljevid_regije.pdf}

\section{Napredna analiza podatkov}

%\includegraphics{../slike/naselja.pdf}

\end{document}
